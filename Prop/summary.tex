% Note: must be in third person, must reflect both ``intellectual
% merit and broader impact'' of the proposal.  ``objectives and
% methods''

%% From the NSF: The proposal must contain a summary of the proposed
%% activity suitable for publication, not more than one page in
%% length.  It should not be an abstract of the proposal, but rather a
%% self-contained description of the activity that would result if the
%% proposal were funded.  The summary should be written in the third
%% person and include a statement of objectives and methods to be
%% employed.  It must clearly address in separate statements (within
%% the one-page summary): (1) the intellectual merit of the proposed
%% activity; and (2) the broader impacts resulting from the proposed
%% activity.  (See Chapter III for further descriptive information on
%% the NSF merit review criteria.)  It should be informative to other
%% persons working in the same or related fields and, insofar as
%% possible, understandable to a scientifically or technically
%% literate lay reader.  Proposals that do not separately address both
%% merit review criteria within the one page Project Summary will be
%% returned without review.

%% State the "objectives", "goals" or "challenges" of this work

\documentclass[12pt]{article}

\topmargin=-0.5in
\oddsidemargin=0in
\evensidemargin=0in
\textwidth=6.5in
\textheight=9.0in

\usepackage{color}
\usepackage{xspace}

% COMMENTS

\newcommand{\comment}[1]{}

\definecolor{pennred}{RGB}{149,0,26}
\definecolor{cornellred}{RGB}{196,18,48}
\definecolor{princetonorange}{RGB}{255,143,0}
\definecolor{tmlblue}{RGB}{0,58,120}  % tml == Toronto Maple Leafs
\definecolor{brownbrown}{RGB}{121,37,0}

\newcommand{\finish}[1]{\ifdraft#1\else\fi}
\newcommand{\dpw}[1]{\finish{\textcolor{tmlblue}{[#1 --DPW]}}} 
\newcommand{\tdm}[1]{\finish{\textcolor{pennred}{[#1 --TDM]}}} 

% Abbreviations

\def\eg{\emph{e.g.\@\xspace}}
\def\ie{\emph{i.e.\@\xspace}}
\def\etc{\emph{etc.\@\xspace}}
\def\etal{\emph{et\ al.\@\xspace}}

% Definitions

\newcommand{\Name}{{\sc MC}\@\xspace}  % Modern Centaur?  MC  
                                       % note there is an old inria system
% called The Centaur System:  https://www.ercim.eu/publication/Ercim_News/enw32/pascual.html

\newcommand{\cd}[1]{\lstinline[backgroundcolor=\color{white}]$#1$}


\begin{document}
\setcounter{page}{1}

 \begin{large}
\begin{center}
SHF:Small:Collaborative Research: \\
Language Support for Resource-Savvy Synthesis
\end{center}
\end{large}

\noindent
\textbf{Overview:}   
In constructing modern software systems, 
programmers must make many implementation decisions, but
the best choices depend upon a complex array of factors that may be
only partially understandable even to excellent programmers.
Examples of such factors include the difficult-to-predict effects of 
parallelization, the shifting demands of dynamic workloads,  the
choice of data representations, and the effects of caching.
In the past, researchers have built effective systems 
for automated synthesis of efficient
implementations such as Hawkin \emph{et al.}'s representation
synthesis, the Stoke super-optimizer, and others.
However, these efforts, while valuable, have typically resulted in 
\emph{one-off}, \emph{stand-alone} systems.

\noindent\textbf{Keywords:} program synthesis; data representation synthesis;
type systems; functional programming; domain-specific languages

\noindent
\textbf{Intellectual Merit:}
In contrast, the goal of our research is to make it possible to integrate
tools for synthesizing high-performance implementations into every-day 
practice and to do so in the context of 
established, modern, typed programming languages.  
We plan to achieve our goal
by designing linguistic mechanisms that allow users to define and use 
their own new \emph{resource-savvy synthesis plug-ins} (\rasps,
pronounced ``rasps'').
Programmers will specify \rasps
using a combination of modules and user-defined
language extensions that enable the
synthesis of efficient implementations, taking into consideration the current
environment and client program.  A basic \rasp will include the definition
of (i) a new abstract interface for clients, (ii) a specification of a
(possibly infinite) space of implementations, (iii)
a cost metric, and (iv) a search strategy.  
%The space of implementations will be specified using typed symbolic 
%values, which will have symbolic types that are constrained using 
%dependent types, and be implemented efficiently using compile-time, 
%type-directed computations.
The space of implementations
will be specified using both \emph{symbolic values} and \emph{symbolic types}.
The latter will define the shapes of data structures, customized for
particular workloads and environments.  To specify the necessary representation 
invariants over such data structures, we propose to augment our
symbolic types with constraints, thereby giving rise to \emph{dependent
symbolic types}.
%will be used to drive
%that are constrained using \emph{dependent types}, and be implemented 
Efficient algorithms for operating over such data structures will be defined
using \emph{compile-time, type-directed computations}.  New 
search strategies may be implemented directly by users, and may range from
brute force enumeration to genetic programming to deep reinforcement learning.
To use a new \rasp, client programs will simply import the \rasp definition
and supply example data to drive the search/synthesis process.
To support performance debugging and analysis of performance vulnerabilities,
we will explore language support for analyzing synthesis 
results and generating candidate inputs that may exhibit unexpected 
performance.  We will implement our ideas in the context of the
Haskell programming language and evaluate the resulting design by testing
it on applications ranging from strictness analysis to
representation synthesis.  We will also study the formal properties of the
type theory resulting from our design.

\noindent
\textbf{Broader Impacts:} 
Language-integrated, resource-savvy synthesis has the potential to
save programmer time while also delivering high-performance implementations.
This proposal, if successful, may have a transformative impact on the
structure, performance and maintainability of modern software systems.
We also propose an educational plan to develop a seminar for 
undergraduate students focused on producing new synthesis-supported,
visual programming languages using Google's Blockly toolkit.  Not only
will the seminars introduce undergraduates to research, they will also
produce new educational tools for students learning to program.



\end{document}


%%% Local Variables:
%%% mode: latex
%%% TeX-master: "proposal.tex"
%%% TeX-PDF-mode: t
%%% End:
